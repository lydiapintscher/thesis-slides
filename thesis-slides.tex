\documentclass{beamer}
\usetheme{Boadilla}
\usecolortheme{beaver}

\usepackage[german]{babel}
\usepackage[utf8x]{inputenc}

\title[Collaborative \& transparent FS development]{Collaborative and transparent Free Software development} 
\author{Lydia Pintscher} 
\institute[KIT]{Institute of Applied Informatics and Formal Description Methods\\
Karlsruhe Institute of Technology
}
\date{\today}

\begin{document}
\begin{frame}
\titlepage
\end{frame}

\section*{\"Ubersicht}

\begin{frame}
\frametitle{\"Ubersicht}
\tableofcontents
\end{frame}

\section{Einleitung}

\begin{frame}
\frametitle{Einleitung}
\begin{itemize}
 \item Freie Software ist heute ein integraler Bestandteil der Technologiewelt
 \item sehr unterschiedliche Projekte mit \"ahnlichen Problemen: Amarok und Halo
 \item mehr Transparenz und Kollaboration
 \item Analyse und Verbesserung des Entwicklungsprozesses mit bekannten Tools
\end{itemize}
\end{frame}

\section{Grundlagen}

\begin{frame}
\frametitle{Grundlagen}
foo
\end{frame}

\begin{frame}
\frametitle{Kollaboration und Transparenz}
foo
\end{frame}

\begin{frame}
\frametitle{Freie Software}
foo
\end{frame}

\begin{frame}
\frametitle{Halo}
\begin{itemize}
 \item Erweiterungen f\"ur Semantic MediaWiki
 \item Vereinfachung und Erweiterung der Nutzung semantischer Daten in einem Wiki
 \item Hauptaugenmerk auf Nutzung im Gesch\"aftsumfeld
 \item sehr starker Einfluss von Hauptsponsor Vulcan Inc.
 \item Atmostph\"are in der Community stark gezeichnet von kommerziellem Einfluss
\end{itemize}
\end{frame}

\begin{frame}
\frametitle{Amarok}
foo
\end{frame}

\section{Analyse des aktuellen Entwicklungsprozesses}

\begin{frame}
\frametitle{Analyse des aktuellen Entwicklungsprozesses}
foo
\end{frame}

\subsection{Halo}

\begin{frame}
\frametitle{Halo}
foo
\end{frame}

\subsection{Amarok}

\begin{frame}
\frametitle{Amarok}
foo
\end{frame}

\subsection{Vergleich und Schlussfolgerung}

\begin{frame}
\frametitle{Vergleich und Schlussfolgerung}
foo
\end{frame}

\section{Design eines verbesserten Entwicklungsprozesses}

\begin{frame}
\frametitle{Design eines verbesserten Entwicklungsprozesses}
foo
\end{frame}

\begin{frame}
\frametitle{Anforderungen, Erwartungen und Rahmenbedingungen}
foo
\end{frame}

\begin{frame}
\frametitle{Kollaborativ in einem Team arbeiten}
foo
\end{frame}

\begin{frame}
\frametitle{Kollaborativ an einer Vision arbeiten}
foo
\end{frame}

\begin{frame}
\frametitle{Kollaborativ eine Roadmap erstellen}
foo
\end{frame}

\begin{frame}
\frametitle{Qualit\"atssicherung}
foo
\end{frame}

\begin{frame}
\frametitle{Bausteine}
foo
\end{frame}

\section{Implementierung eines verbesserten Entwicklungsprozesses}

\begin{frame}
\frametitle{Implementierung eines verbesserten Entwicklungsprozesses}
foo
\end{frame}

\begin{frame}
\frametitle{Einf\"uhrungsszenario}
foo
\end{frame}

\begin{frame}
\frametitle{Kommunikation und Kollaboration in einem verteilten Team}
foo
\end{frame}

\begin{frame}
\frametitle{Kollaborativ an einer Vision arbeiten}
foo
\end{frame}

\begin{frame}
\frametitle{Kollaborativ eine Roadmap erstellen}
foo
\end{frame}

\begin{frame}
\frametitle{Qualit\"atssicherung}
foo
\end{frame}

\section{Evaluation}

\begin{frame}
\frametitle{Evaluation}
foo
\end{frame}

\begin{frame}
\frametitle{Umfrage}
foo
\end{frame}

\begin{frame}
\frametitle{Ver\"anderung in der Offenheit des Entwicklungsprozesses}
foo
\end{frame}

\section{Zusammenfassung und Ausblick}

\begin{frame}
\frametitle{Zusammenfassung und Ausblick}
foo
\end{frame}

\end{document}
